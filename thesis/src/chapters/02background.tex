\chapter{Background}
\label{cha:background}

\section{Clustering}
Clustering is an unsupervised data analysis technique that can be informally defined as the partitioning of a set of objects into groups called clusters; such partitioning must ensure that objects within a same cluster are more similar\footnote{Provided that a binary ordering relation is defined on the set of objects.} to each other than to objects in other clusters. \cite{clustering}

Clustering is widely employed in numerous scenarios that require to group together similar data points, or to extract knowledge from a set of objects in the absence of any prior information. For instance, it is used in marketing and finance as a profiling tool; % ref: TODO
in image processing and computer vision, as a segmentation technique \cite{Gonzalez2017}, where it plays a pivotal role in various fields, such as remote image sensing \cite{remote-sensing} and digital forensics \cite{forensics}; % ref: gonzalez image processing capitolo su segmentation
by energy distribution companies, to optimize the allocation of resources to end users. % ref: TODO

\section{Basics of quantum computing}

\section{Quantum annealing}

\section{Neutral atoms technology}