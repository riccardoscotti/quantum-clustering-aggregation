\chapter{Background}
\label{cha:background}

\section{Clustering}
A clustering problem is defined as the task of partitioning a set of objects, for which there exists a similarity metrics of some sort, into multiple, nonempty sets called clusters. The primary goal of clustering is to ensure that objects within the same cluster exhibit a high degree of similarity, while objects in different clusters are significantly distinct from one another.

Clustering is widely employed in numerous scenarios that require to group together similar data points or objects. For instance, it is used in marketing and finance as a profiling tool; % ref: TODO
in image processing and computer vision, as a segmentation technique, which is ubiquitous in industrial settings as well as in remote image sensing; % ref: gonzalez image processing capitolo su segmentation
by energy distribution companies, to optimize the allocation of resources to end users. % ref: TODO

\section{Basics of quantum computing}

\section{Quantum annealing}

\section{Neutral atoms technology}